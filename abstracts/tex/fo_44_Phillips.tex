
    \begin{abstract_online}{Extracting and Registering References to Improve Scholarly Impact of ETDs}{%
        Mark .E. Phillips$^{1}$,  Hannah Tarver$^{1}$,  Kristy Phillips$^{1}$, \underline}{%
        }{%
        $^1$ University of North Texas USA
            }
        
        Backgroun:
        
        The UNT Libraries began registering ETDs bjects Identifiers (DOIs) to the documents submitted in 2023. This decision was made to try and position our students’ ETDs in the best light and to collaborate better in the scholarly communication landscape. Assigning DOIs to ETDs is not a new activity for many institutions and has been a major way of providing persistent linking to these resources and improving their discovery ability. The assignment of DOIs introduces the option to register references to the scholarly resources cited by the work receiving the DOI in the Crossref platform. Doing so allows for Crossref to display interconnections of cited works and improves their scholarly impact. While assigning DOIs to ETDs is common, registering references for these resources is much less common. Currently, there are over 660,000 theses and dissertations registered with Crossref. Of these, just 1,558 (less than 0.24%) have references registered with their DOI and those records come from only 19 institutions.While there have been many previous research projects into the automatic identification and extraction of references in scholarly publications -- including ETDs -- this research does not seem to have made its way into ETD workflows for most institutions tasked with their long-term hosting. One likely reason is that references in ETDs can take many formats and occur in different places within the document. These references may be arranged using different citation styles. In some cases a single ETD could exhibit multiple references sections, with multiple reference formats, e.g., alphabetical vs. numbered, with multiple citation styles. This makes automated extraction of references challenging.ObjectivesThis paper presents work by the UNT Libraries to build reference lists for ETDs created by students at UNT for two main purposes. First, these lists provide references to Crossref to support additional visibility and functionality related to DOIs created for ETDs in the UNT collections. Second, they provide a building block for local collection analysis regarding high-level usage of library resources purchased for our students and faculty.
        
        Method:
        
        In order to register our references with Crossref, we decided to create a human-curated workflow to extract text references from ETDs. First we established a workflow with documentation for student assistants in the libraries to identify and copy references from the submitted PDF documents. Next we adopted a simple text format to serialize each text reference from the ETD. We developed simple normalization and transformation scripts in Python to modify the copied text reference so that it is presented in a standard format. Finally, these text-based references lists are added to a public Git repository on Github.com where it can be used as a beginning point for different analysis and data mining.
        
        Results:
        
        In 2023, 410 ETDs were added to the UNT Libraries collection, containing nearly 48,000 references. Once we had these references in a consistent format we were able to leverage the Crossref API to register these references with our DOIs. The goal for including these references is to increase the scholarly impact of our students' theses and dissertations while also contributing to the global scholarly communication landscape in a new way.
        
        Conclusio:
        
        This paper presents a background of automated methods for reference extraction in ETDs and also discusses the benefits and challenges of building reference lists for these documents in our collections. It presents the workflow that has been piloted at the UNT Libraries to extract and normalize references for the 410 ETDs submitted in 2023 along with the costs associated with human efforts to prepare these reference lists. The goal of the paper is to provide a clear set of steps and methods that other institutions could replicate without complicated technologies or processes.
    \end{abstract_online}
    
