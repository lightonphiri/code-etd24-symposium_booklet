
    \begin{abstract_online}{Do More Complete Dissertations’ Metadata Get More Engagement?}{%
        Behrooz B. Rasuli$^{1}$, Michael Boock$^{2}$, Joachim Schöpfel$^{3}$, \underline{Brenda Van Wyk}$^{4}$}{%
        }{%
        $^1$ Iranian Research Institute For Information Science And Technology(IranDoc) , Iran\newline{}$^2$ Oregon State University, USA,$^3$University Of Lille, France, $^4$University Of Pretoria, South Africa}
            }
        This study investigates the role of metadata quality in Electronic Theses and Dissertations (ETDs), focusing on its completeness and its impact on discoverability and user engagement within institutional repositories (IRs). Using DSpace@MIT as a case study, the current research analyzed 22,276 doctoral dissertations to assess metadata completeness and its correlation with the number of views and downloads. Various metadata fields and usage statistics were extracted for detailed analysis. The study identified a moderate positive correlation between the number of unique metadata fields and both the Department Views Ratio (DVR) and Department Download Ratio (DDR), suggesting that enhanced metadata can improve the visibility and accessibility of dissertations. Additionally, the length of abstracts is positively correlated with engagement metrics. In contrast, title length does not significantly influence the visibility. These findings showed the importance of high-quality metadata in enhancing the discoverability of ETDs.
    \end{abstract_online}
    
