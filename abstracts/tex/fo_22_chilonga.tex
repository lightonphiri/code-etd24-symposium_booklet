
    % \begin{abstract_online}{Do More Complete Dissertations’ Metadata Get More Engagement?}{%
    %     Behrooz B. Rasuli$^{1}$, Michael Boock$^{2}$, Joachim Schöpfel$^{3}$, \underline{Brenda Van Wyk}$^{4}$}{%
    %     }{%
    %     $^1$ Iranian Research Institute For Information Science And Technology(IranDoc) , Iran\newline{}$^2$ Oregon State University, USA,$^3$University Of Lille, France, $^4$University Of Pretoria, South Africa}
    %     This study investigates the role of metadata quality in Electronic Theses and Dissertations (ETDs), focusing on its completeness and its impact on discoverability and user engagement within institutional repositories (IRs). Using DSpace@MIT as a case study, the current research analyzed 22,276 doctoral dissertations to assess metadata completeness and its correlation with the number of views and downloads. Various metadata fields and usage statistics were extracted for detailed analysis. The study identified a moderate positive correlation between the number of unique metadata fields and both the Department Views Ratio (DVR) and Department Download Ratio (DDR), suggesting that enhanced metadata can improve the visibility and accessibility of dissertations. Additionally, the length of abstracts is positively correlated with engagement metrics. In contrast, title length does not significantly influence the visibility. These findings showed the importance of high-quality metadata in enhancing the discoverability of ETDs.
    % \end{abstract_online}
    


    \begin{abstract}{Examining the Cultural and Institutional Factors Impacting ETD Visibility in Zambia: Policy and Practice Implications}{%
        Mpundu Chilonga$^{1}$, Cecilia Kasonde$^{1}$, Chewe Mumba$^{1}$}{%
        $^1$ Kwame Nkrumah University, Zambia}{%
        }
        This study delves into the cultural and institutional factors influencing the visibility of Electronic Theses and Dissertations (ETDs) in Zambia and their implications for policy and practice. Employing a mixed-methods research design comprising surveys, interviews, and document analysis, the research seeks to illuminate the multifaceted landscape surrounding ETD accessibility. The findings of this study unveil a nuanced understanding of the challenges and opportunities in the Zambian context. They reveal a complex interplay of cultural attitudes towards digital scholarship, institutional infrastructures, technological literacy, and policy frameworks shaping the accessibility of ETDs. Furthermore, the study uncovers significant barriers hindering the dissemination and utilization of ETDs beyond academia. These barriers include limited digital infrastructure, inadequate institutional support, and prevailing cultural norms favouring traditional forms of knowledge dissemination. The implications of these findings underscore the need for targeted policy interventions and institutional reforms aimed at fostering a culture of open access and digital scholarship in Zambia. By addressing these cultural and institutional challenges, Zambia can harness the full potential of ETDs as valuable resources for education, research, and socioeconomic development.KEYWORDS: Digital Repositories, Resource accessibility, Research visibility

    \end{abstract}
    
