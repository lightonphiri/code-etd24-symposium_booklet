
    \begin{abstract_online}{Examining the Cultural and Institutional Factors Impacting ETD Visibility in Zambia: Policy and Practice Implications}{%
        Mpundu Chilonga$^{1}$, Cecilia Kasonde$^{1}$, Chewe Mumba$^{1}$}{%
        }{%
        $^1$ Kwame Nkrumah University, Zambia}{%
        }
        This study delves into the cultural and institutional factors influencing the visibility of Electronic Theses and Dissertations (ETDs) in Zambia and their implications for policy and practice. Employing a mixed-methods research design comprising surveys, interviews, and document analysis, the research seeks to illuminate the multifaceted landscape surrounding ETD accessibility. The findings of this study unveil a nuanced understanding of the challenges and opportunities in the Zambian context. They reveal a complex interplay of cultural attitudes towards digital scholarship, institutional infrastructures, technological literacy, and policy frameworks shaping the accessibility of ETDs. Furthermore, the study uncovers significant barriers hindering the dissemination and utilization of ETDs beyond academia. These barriers include limited digital infrastructure, inadequate institutional support, and prevailing cultural norms favouring traditional forms of knowledge dissemination. The implications of these findings underscore the need for targeted policy interventions and institutional reforms aimed at fostering a culture of open access and digital scholarship in Zambia. By addressing these cultural and institutional challenges, Zambia can harness the full potential of ETDs as valuable resources for education, research, and socioeconomic development.KEYWORDS: Digital Repositories, Resource accessibility, Research visibility

    \end{abstract_online}
    
