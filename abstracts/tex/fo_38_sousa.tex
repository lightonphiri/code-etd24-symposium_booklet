
    \begin{abstract_online}{Characterization of scientific production on electronic theses and dissertations based on a bibliometric analysis}{%
      Juliana Sousa$^{1}$ , \underline{ Tainá Assis}$^{1}$}{%
        }{%
      $^1$ Instituto Brasileiro de Informação em Ciência e Tecnologia, Brazil     \newline}
        
        
        }
        The text presents the importance of expanding the dissemination of electronic theses and dissertations (ETD). Given this importance, the study proposes a bibliometric analysis of scientific production that focuses on ETD. The time frame used for the analysis was made between 2002 and 2023. To assist in data recovery, the Publish or Perish tool was used. Therefore, the result was the identification of 601 documents, of which 61 were selected for analysis. There was a greater maturity in scientific publications in English, with the majority of publications. Given this, it is possible to infer that there are still gaps in the topic to be filled, which could lead to greater incentives for studies in the area.

        \end{abstract}
        
