
        \begin{abstract}{DataCite Connect: Global visibility for your ETDs and beyond with PIDs}{%
            Gabriela Mejias}{%
            DataCite, Germany}{%
            }
        PIDs and their associated metadata can be considered the building blocks of research infrastructure (Meadows et al., 2019).  In addition,  PIDs play a central role in the Open Science framework as they can increase transparency and recognition in research, and facilitate integration and interoperability, as well as making research FAIR (Wilkinson et al. 2016).

        This half day  workshop will explore how Persistent Identifiers (PIDs), particularly Digital Object Identifiers (DOIs), can significantly boost the global visibility of Electronic Theses and Dissertations (ETDs) and other research outputs. We'll cover the basics of PIDs, showing how DOIs improve visibility and discoverability for academic work. Through case studies, you'll see real-world examples of research organizations that have successfully integrated PIDs into their ETD repositories and other workflows. The session culminates in a breakout discussion to share best practices and explore implementation strategies. This workshop offers valuable insights for enhancing the visibility of your research through DataCite's PIDs, providing a unique opportunity to connect with the broader research community.
        Proposed activities:

        - (20 min) Welcome and Ice breaker activity
        - (30 min) Introduction to Persistent Identifiers (PIDs) and how they support open research
        - (30 min) Discover the role of Digital Object Identifiers (DOIs) in making ETDs more visible and discoverable on a global scale.
        - (30 min) Improving ETD and other research workflows with PIDs.
        - (20 min) Coffee break
        - (40 min) Case studies featuring invited guest speakers from research organizations implementing PIDs in their ETDs, research data repositories (and other!) workflows
        - (50 min) Breakout session to discuss best practices and implementation workflows
        - (10 min) Adjourn

        Details of similar previous workshops: https://datacite.org/event/datacite-connect-gothenburg-dataciteconnect23/
        https://datacite.org/event/datacite-connect-buenos-aires-dataciteconnect23/
        ETD \& Open Science: Maximizing the Discoverability and Impact
        through Persistent Identifiers (PIDs) https://etd2023.inflibnet.ac.in/programme.php

        Facilitators: Gabriela Mejias, Bosun Obileye (DataCite). Guest speakers from research institutions will be confirmed upon workshop acceptance notification.
        \end{abstract}
        
