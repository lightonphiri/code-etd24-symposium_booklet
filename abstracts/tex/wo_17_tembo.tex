
    \begin{abstract_online}{Leveraging ORCID's Global Participation Program and Regional Consortium Approach to enhance Global  ETD discoverability and reuse}{%
        \underline{Lombe Tembo}$^{1}$}{%
        }{%
        $^1$ ORCID, Zambia}
        This workshop proposal aims to explore strategies for leveraging ORCID and Persistent Identifiers (PIDs) to enhance the discoverability and reusability of Electronic Theses and Dissertations (ETDs) on a global scale. ETDs represent a vital component of scholarly communication, providing valuable insights and contributions to research across various disciplines. However, their full potential often remains untapped due to challenges related to visibility, accessibility, and interoperability.

        ORCID, providing an open registry and a unique identifier for researchers, offers a transformative solution to address these challenges by placing researchers at the center of scholarly communication. With over 6000 connected systems worldwide, ORCID provides a robust infrastructure to facilitate trusted data exchange through  integrations with global scholarly systems and indexes. By adopting ORCID in ETD repositories, institutions can enhance the quality of data associated with ETDs, improve interoperability with other scholarly systems, and increase visibility and reuse on a global scale.

        In addition to ORCID, other Persistent Identifiers (PIDs) play a crucial role in enhancing the discoverability of ETDs by providing unique and persistent links to scholarly output. Through the implementation of PIDs in ETD repositories and connected systems, institutions can ensure that ETDs remain discoverable and accessible over time, thus maximizing their impact and visibility within the research community.

        Furthermore, this workshop will explore the role of ORCID's Global Participation Program (GPP) and Regional Consortium initiatives to facilitate global communities of practice, fostering collaboration among diverse stakeholders that advance ETD management practices. By participating in the GPP and regional consortia, institutions can leverage collective expertise, resources, and networks to enhance the visibility and impact of ETDs on a global scale.

        Through collaborative discussions, participants will gain actionable insights and strategies to effectively leverage ORCID and PIDs for ETD management and dissemination. By engaging with this workshop, participants will be empowered to maximize the impact of ETDs through PIDs within their institutions and contribute to the advancement of scholarly communication on a global scale.

        This session can be part of a wider workshop relating to scholarly infrastructure and/or PIDs, and can also be placed within the main conference presentation line-up.
    
    \end{abstract_online}
    
