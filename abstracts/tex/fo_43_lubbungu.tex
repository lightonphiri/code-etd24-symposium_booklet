
    \begin{abstract_online}{E-Theses and Dissertations in Zambia: A Case Study of Two Universities in Kabwe }{%
        \underline{Jive Lubbungu}$^{1}$, Chewe Mumba$^{1}$}{%
        }{%
        $^1$ Kwame Nkrumah University, Zambia}
            }
        This study investigates the successes and challenges in the implementation of Electronic Theses and Dissertations (ETDs) at Kwame Nkrumah and Mulungushi universities in Kabwe district, Zambia. Employing a qualitative research approach, data were collected from four purposively selected key informants using structured interview guides. These informants were personnel from the e-resources departments of the university libraries. The findings reveal that both institutions share common attributes, such as the establishment of institutional repositories and the successful initial implementation of ETDs. However, the study identifies significant challenges including staff resistance to depositing their dissertations into institutional repositories, a lack of expertise in configuring the institutional repository platforms, and intermittent network connectivity. Thematic analysis was utilized to analyze the data. The study concludes that while some progress has been made in the implementation of ETDs, the current state at the two institutions has not yet reached the desired level. To address these issues, the study recommends the following: enhanced training programs for staff on the importance and use of ETDs, improved technical support and infrastructure for repository management, and strategies to foster a culture of compliance and participation among academic staff.
    
    \end{abstract_online}
    
