
    \begin{abstract_online}{Resilience and ETD Repositories in Poor Countries}{%
        \underline{Hussein Suleman}$^{1}$}{%
        }{%
        $^1$ University of Cape Town, South Africa\newline{}$^1$ Networked Digital Library of Theses and Dissertations, United States}
        ETD repositories are quite common in countries and universities around the world and are often among the first digital repositories to be created.  In poor countries, however, ETD repositories face many challenges because of the interplay between socio-economic and technical factors.  Some past African ETD repositories no longer exist, with the theses they contained being no longer accessible online.  This talk will focus on approaches to mitigate this risk by designing repositories that are resilient and robust in the face of disaster.  Many years of experiments with principled design and simple architectures has demonstrated that the inherent complexity in modern repository tools can be avoided altogether.  These alternative technical solutions can address how we build systems to safeguard our intellectual assets in the future, while also informing how the whole world designs resilient systems.
    
    \end{abstract_online}
    
