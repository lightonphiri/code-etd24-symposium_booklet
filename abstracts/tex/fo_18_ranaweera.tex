
    \begin{abstract_online}{Nurturing Advanced Research Culture among Medical Practitioners through ETDs: A case study from University of Kelaniya, Sri Lanka}{%
        \underline{Lanka Ranaweera}$^{1}$,}{%
        }{%
        $^1$ Faculty Of Medicine, University Of Kelaniya}
            }
   Introduction: In the context of the Faculty of Medicine at the University of Kelaniya, Sri Lanka, electronic theses and dissertations (ETDs) play a pivotal role in fostering an advanced research culture among medical practitioners. This study delineates the multifaceted contributions of ETDs across various academic programs, including the Master of Public Health (MPH), BSc Degree in Speech and Hearing Sciences, BSc Degree in Occupational Therapy, and PhD dissertations from the Molecular Medicine Unit.
   
   Purpose of the Present Study: The aims of the present study were to highlight the significance of ETDs in advancing medical research data and showcase the diverse applications of ETDs across different academic programs at the Faculty of Medicine.
   
   Methodology: A comprehensive review of ETDs produced by students across various academic programs within the Faculty of Medicine at the University of Kelaniya was conducted. The review encompassed ETDs from the MPH program, BSc Degree programs in Speech and Hearing Sciences and Occupational Therapy, and PhD dissertations from the Molecular Medicine Unit. Data were analyzed to identify key themes, contributions, and impacts of ETDs on healthcare services and research activities.The Master of Public Health (MPH) program equips students with evidence-based approaches to promote health and prevent diseases, with ETDs contributing substantially to evidence-based public health interventions and policy formulation. PhD dissertations from the Molecular Medicine Unit provide consultancy services in Molecular Diagnosis and DNA typing and engage in research on infectious/vector-borne diseases and genetic diseases, thereby enhancing healthcare services in Sri Lanka. Importantly, it was noticed that the students of the BSc Degree program in Speech and Hearing Sciences produce compact discs (CDs) rather than ETDs. The BSc Degree program in Occupational Therapy commenced only two years ago and has not yet produced any theses.
   
   Discussion: The findings from this research highlight the significant contributions of various academic programs to the field of public health and healthcare services in Sri Lanka. Each program plays a crucial role in advancing health-related knowledge, practices, and policies. The MPH program emphasizes evidence-based approaches to health promotion and disease prevention, equipping students with critical skills. Their ETDs contribute substantially to public health interventions and policy formulation, enhancing the effectiveness of interventions and policies at both local and national levels. PhD dissertations from the Molecular Medicine Unit illustrate the program's pivotal role in advancing healthcare services through specialized research and consultancy, directly enhancing healthcare delivery. In contrast, the BSc Degree program in Speech and Hearing Sciences has been producing CDs rather than ETDs, hindering accessibility and dissemination of research findings. Transitioning to ETDs could facilitate interdisciplinary research and collaboration. The BSc Degree program in Occupational Therapy, established only two years ago, has not yet produced any theses. Adopting the practice of producing ETDs from the outset would ensure that student research contributes to the global body of knowledge in occupational therapy and aids in developing evidence-based practices. ETDs serve as catalysts in identifying potential opportunities for collaborative research endeavors, driving innovative solutions despite financial constraints.
   
   Conclusion:It is imperative to encourage the initiation of ETD production for newly established degree programs that have not yet commenced research activities. Transitioning from CDs to ETDs is highly recommended to amplify the impact and accessibility of student research outputs.The widespread adoption of ETDs in all Faculty of Medicine programs is crucial for promoting interdisciplinary research, fostering partnerships, and innovating health  care solution.
    \end{abstract_online}
    
