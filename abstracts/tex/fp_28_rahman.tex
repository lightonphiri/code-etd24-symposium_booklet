
        \begin{abstract}{ Determining the factors influencing the utilization of open source digital repository software in the preservation of ETDs in academic libraries in Bangladesh}{% MDr. Md. Zillur Rahman }{%
              Ahsanullah University of Science and Technology* {%
            }
 Problem/motivation/goal:This article aims to identify the variables that affect the use of open source digital repository software (DRS) for ETD preservation in academic libraries of Bangladesh. Additionally, the study examines the compliance difficulties with standards and protocols, such as metadata, WebDAV, OpenSearch, OpenURL, RSS, ATOM, OpenArchives Initiative (OAI)-PMH, OAI-ORE, SWORD, and WebDAV, for access, ingest, and export while using open source DRS. The link between technical and financial aspects of employing DRS is also covered in the study. Methodology/approach:The current investigation would employ a quantitative methodology. Academic librarians in Bangladesh will be given a structured questionnaire to complete in order to meet the study goals. The instrument will be shared on a number of forums, email groups, and mailing lists for libraries. There will be three portions to the instrument (A, B, and C). The respondents' demographic information is in Section A. The questions in Section B dealt with aspects of employing DRS to preserve ETDs. Questions about complaints about employing DRS will be found in Section C. The degree of each question will be determined using a 7-point Likert scale. Descriptive statistics will be used to examine the gathered data. A statistical significance test will be utilized in addition to descriptive statistics to examine the connection between technical and economic factors of using DRS. ObjectivesThe main objective of the present study is to explore the factors and usage of open source DRS. 
	
	The present study attempts to ascertain the following objectives:
	
	1.To determine the factor of using DRS in the preservation of ETD’s.2.To know the complaints about using DRS for the preservation of ETDs.Research QuestionsRQ1. What is the level of factors influencing the usage of open source DRS by academiclibraries in Bangladesh?
	2. What is the relationship between the variables, multiple factors, and the usage of open source DRS?RQ3. What are the complaint issues that are adhered to by the academic institutions in Bangladesh?
	
	Anticipated results:Despite the advantages of open source DRS of capturing and ingest of ETDs, including metadata about materials, easy access to the ETDs both by listing and searching and long-term preservation of the ETDs, it has some drawback too which are- flat file and metadata structure, poor user interface, lack of scalability and extensibility, limited API, limited metadata features, limited reporting capabilities and lack of support for linked data. Overcoming all the barriers the use of open source DRS has been increasing day by day in Bangladesh. The majority of open-source DRS are functional in nature, and their base URL is OAI-PMH. DSpace continues to be the program of choice for DRS content management. The majority of DRS do not have rules in place for content control. The majority of DRS do not provide usage data, and English continues to be a priority language for the material scattered among them. Web 2.0 tools are included in a considerable number of IRs, with RSS being the most used Web 2.0 tool. Sizable portions of IRs have not altered their user interface. The interface of the majority of IRs is bilingual.According to Crow (2002) and Rahman (2015), DRS in Bangladesh essentially began as a portion of the institution's digital repository, which is defined as "[...] a digital archive of the intellectual product created by the faculty, research staff, and students of an institution and accessible to end-users both within and outside of the institution, with few if any barriers to access." Open source software was chosen by academic libraries for their institutional repositories and comprehensive digital library platforms. In Bangladeshi libraries, computerization and digitalization have gained significant traction in the last twenty years, which is an amazing accomplishment (Rahman et al, 2015).
      

        \end{abstract}
        
