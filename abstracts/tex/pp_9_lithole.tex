\begin{abstract}{The University of Johannesburg’s journey to enhance the content of the Institutional Repository (IR) and improve discoverability}{%
    Mutali M Lithole}{%
    University of Johannesburg}{%
}

Institutional Repositories has been well established in most academic libraries over the past few decades. As technology developed in all spheres of life, the same development could be seen in software and technology used in the IR environment. We are all aware of software systems such as DSpace, EPrints, and Fedora, to name but a few (Castagne, 2013). The UJ library established their IR in 2008 and currently consists of research output generated by the university which includes journal articles, conference proceedings, books, and book chapters. In the last 14 years, it has been a journey during which the library used and experimented with several systems to enhance the content of the IR and to increase visibility. These systems include: DSpace, Vital, and most recently, Esploro. The purpose of this paper is to share our journey and experiences with these systems and software used over the past few years. The presentation will start with a short history of our IR journey. This will be followed by a more technical discussion, not only on the systems used to store and access the content but also the system we use to collate the relevant content. Towards the end of the presentation, the focus will be on the latest system UJ implemented, namely Exlibris Esploro, which allows us to maximize the impact of our institutional research using intelligent data capturing, thereby reducing the workload of the IR staff and also allowing increased visibility of the content as well as the authors who produced the content.

Keywords: Open access, institutional repositories, Esploro, University of Johannesburg. Further reading: \url{https://www.usetda.org/resources/institutional-repository-software/}

\end{abstract}

        
