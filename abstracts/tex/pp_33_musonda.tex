
       \begin{abstract}{Enhancing Electronic Thesis and Dissertation Visibility: A Focus on Institutional Repository Platforms and Inherent Challenges in the Nigerian Context}{%
    David, Okhakhu O}{%
    The University Of Zambia%
}

Higher Education Institutions (HEIs) regularly publish manuscripts of academic research that provide useful insights into social, economic, and technological issues affecting society (Phiri and M’sendo, n.d.). In Electronic Theses and Dissertations (ETDs), researchers often focus on impact-driven research with the potential to inform policy direction. However, due to the large size of these ETD manuscripts, important stakeholders, such as local mainstream media outlets, find it difficult to synthesize the content of these manuscripts. A potential solution would be to leverage advances in Artificial Intelligence by using Natural Language Processing (NLP) techniques to summarize the ETDs. As Allahyari defines it, automatic text summarization is the task of producing a concise and fluent summary while preserving key information content and overall meaning. Text summarization techniques can generate snippets of scholarly research output that are concise and easily understood by non-technical persons (Allahyari et al., 2017; Ingram et al., n.d.).

Existing literature broadly categorizes automatic summarization into two classes: abstractive summarization and extractive summarization. This study seeks to understand the challenges with synthesizing ETDs and to design and implement software tools to automatically summarize ETDs, focusing on mining text data and creating tools that allow automated summarization and text modification.

Specifically, the objectives of this study are:
1. To determine how frequently research findings are reported in mainstream media.
2. To investigate challenges in synthesizing long documents (ETDs).
3. To implement summarization models for summarizing ETDs for public consumption.
4. To evaluate the summarization models.

The proposed research methodology is as follows:
- Frequency of Publishing Research Findings: Manual and automatic content analysis of existing media publications will be conducted to determine the frequency of publishing research findings.
- Challenges Synthesizing ETDs: Interviews will be conducted with purposively sampled journalists from randomly selected media outlets.
- Design and Implementation of ETD Summarization Models: Classic text summarization techniques, such as abstractive and extractive summarization, will be employed to build the ETD summarization models. Additionally, publicly available ETDs from HEIs in Zambia will be used to construct a dataset for the study.
- Evaluation of ETD Summarization Models: Standard evaluation metrics for text summarization, such as ROUGE (Johnson, n.d.), BLEU (Allahyari et al., 2017; Johnson, n.d.), BERTScore (Bhandari et al., 2020; Zhang et al., n.d.), and METEOR (Ermakova et al., 2019; Johnson, n.d.), will be used to assess the effectiveness of the ETD summarization models. Additionally, human evaluation will be employed to determine the perceived usefulness of the ETD summarization models.

This study could provide valuable insights into the challenges of enabling the general public to engage with ETDs and how the application of NLP and automatic text summarization can help overcome some of these challenges.

\end{abstract}

