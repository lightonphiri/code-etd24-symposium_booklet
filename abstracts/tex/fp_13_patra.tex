
        \begin{abstract}{Global Visibility of National ETD Repositories of G20 Countries: comparative studies with respect to NDLTD’s Meta repository}{% Sukanta Kumar Patra }{%
            Vidyasagar College for Women  {%
            }
   ETDs are primary information sources, that originate from doctorial theses or dissertation submitted to university for the doctorial award. The FAIR ETDs are those electronic theses and dissertations, which are findability, accessibility, interoperability, and reusability (FAIR). The study should be cover to ETD initiatives by G20 members countries, comprises 19 countries (Argentina, Australia, Brazil, Canada, China, France, Germany, India, Indonesia, Italy, Japan, Republic of Korea, Mexico, Russia, Saudi Arabia, South Africa, Turkey, United Kingdom, and United States) and the European Union. The G20 member’s countries represent around 85% of the global GDP, over 75% of the global trade, and about two-thirds of the world population. This study aims to analyze the current state of the ETD repositories of the G20 member countries and to describe their characteristics and performance in brief. The major objectives of the study are to analyze the importance of ETD in global context, to find out linkage between Global ETD and ETDs of G20 member’s Countries and comparative scenario of ETD initiatives with respected to NDLTD repositories. By taking in to considerations the findings, at the end of the study, there are some proposals of recommendations to further improve the status of national ETD repositories.
      

        \end{abstract}
        
