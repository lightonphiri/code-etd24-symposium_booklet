
    \begin{abstract_online}{Unlocking the Potential of ETDs: Implementation Novel ETD Repository in Chartered Institute of Personnel Management in Sri Lanka}{%
        \underline{Kamani Perera}$^{1}$, G. Weerathunga K$^{1}$, Ravindu Sachintha$^{1}$,
        }{%
        $^1$ Chartered Institute Of Personnel Management, Sri lanka\newline{}$^2$ indika@cipmlk.org, Sri lanka}
            }
	Introduction;
	
	Electronic Theses and Dissertations (ETDs) represent a vital and widely utilized open-access resource for scholars and researchers globally. Within higher education, these digital versions of theses and dissertations serve as crucial reservoirs of information and knowledge for forthcoming research activities. ETDs cover both electronic versions and traditional hard copies of scholarly works submitted by researchers to their respective universities or institutions. In brief, ETD can be defined as any thesis or dissertation mainly submitted, archived, and disseminated in electronic format. The creation of ETD repositories represents a revolutionary advancement in utilizing Information and Communication Technology (ICT) to organize institutional research materials systematically, nurturing a creative platform to drive forthcoming research activities. However, in Sri Lanka, despite the growing significance of ETDs, there exists a gap in the availability and accessibility of these resources, particularly in the field of Human Resource Management (HRM). The implementation of a novel ETD repository tailored specifically for HRM research in Sri Lanka.
	
	Objectives;
	
	To implement novel ETD repository to serve the scholarly community to conduct novel researchTo increase the accessibility and visibility of HRM research outputs by providing a centralized platformTo enhance the research culture and academic reputation of Sri Lanka by promoting open access to high-quality HRM scholarship through the ETD repository
	
	Methods;
	
	Implementing new ETD repository has become a crucial task and initiatives has already taken to upload the selected ETDs from bachelor and master level scholars. DSpace software is being used to build up the repository according to the customary needs. Copyright clearance has duly taken to host the institutional ETDs on the platform. Open access ETDs which collected through web navigation are uploaded to the repository based on the institutional curriculum. Initially, ETDs are uploaded to the repository in descended order starting from 2023 the latest. Anti-plagiarism software is being used to check the plagiarism issues in ETDs before uploading to the repository.. Metadata can be considered as an integral part of ETDs lifecycle. Thus, it is ensured to adhere internationally recognized metadata standards such as ETD-MS, established by the Networked Digital Library of Theses and Dissertations (NDLTD).Results and 
	
	Conclusion;
	
	Quality ETDs are selected by the appointed academic committee. Moreover, proper guidance are provided to the newly enrolled students and encourage them to make use the institutional repository and come up with innovative research titles for their academic assignments. Regular awareness programmes are conducted to make the optimum use of the ETD repository and statistics are collected for annual evaluations and scholars can make comments for further improvement of the repository. This is the starting point of the scholarly journey by way of unlocking the potential of ETDs and inviting the new generation to conduct novel research using ETD repository and find out solutions for the burning questions in the field of HRM in the country. Moreover, this ETD repository links with Google Scholar and thereby generate Google Scholar Ranking for the research scholars who are the authors of the particular ETDs and Webometric Ranking for the hosting institute of the repository. It provides global visibility to the HRM research. It preserves valuable resources and thereby reduce the dependance on physical copies and enables efficient search and retrieval of information. Further, ETDs contribute environmental sustainability by decreasing paper consumption. In this context, ETD repositories create a dynamic and vibrant knowledge-sharing eco system that empowers the scholarly community to thrive in their careers and contribute meaningfully to build vibrant research culture.s.
    \end{abstract_online}
    
