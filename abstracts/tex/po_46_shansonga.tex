
    \begin{abstract_online}{Automatic Electronic Thesis and Dissertation Guideline Verification For Consistently Formatted Manuscripts }{%
        \underline{Mubanga C. Chibesa}}$^{1}$,{Albertina Mooka, }}$^{1}$,{I}}$^{1}$, { Gift Muwele}}$^{1}$, {Lwiime Shansonga}}, Lighton Phiri,$^{1}$}{%
        }{%
        $^1$ The University Of Zambia }
      
      Introduction: Higher Education Institutions worldwide enforce guidelines and academic approaches to ensure scholarly integrity and adherence to academic standards(Razı et al., 2019).The University of Zambia, is not an exception. Just like most HEIs it offers  training to postgraduate students and one of the key aspects of postgraduate training is producing an Electronic Thesis and Dissertation manuscript. The Directorate of Research and Graduate Studies (DRGS) at the University of Zambia provides guidelines which stipulate how ETD’s should be formatted. However, the process of  checking for conformance is a manual and tedious procedure, resulting in submission of inconsistently formatted manuscripts in the Institutional Repository (IR). To address this challenge our  project seeks to implement a tool that will automate the process of checking ETD’s compliance against established postgraduate guidelines. The tool will leverage data mining techniques to perform these tasks. More specifically, Document Layout Analysis (DLA) (Binmakhashen & Mahmoud, 2019) will be the core approach used in  the implementation. The tool will flag off portions of ETD manuscripts that do not  conform to established guidelines. Hence, this will help resolve the inconsistencies in the format of submitted manuscripts.
      
      Methodology:Using a mixed methods approach, document analysis will be employed to understand the postgraduate guidelines stipulated in the “Regulations and Guidelines for Postgraduate Studies” guidelines document; content analysis will be used on randomly sampled ETDs in order to experimentally determine the extent of the problem and, finally, a DLA Natural Language Processing model will be developed and evaluated using standard DLA metrics such as.Structure Similarity Index and Intersection over Union.
    
    Results:This study is part of ongoing work aimed at developing an automated tool that will be verifying ETD manuscripts compliance against postgraduate guidelines. Upon successful completion of this project, we anticipate a reduced workload associated with the process of manually checking if manuscripts conform to postgraduate guidelines, freeing up time and resources to handle other academic tasks like attending more to the content of the manuscripts. Not only that, academic integrity is consequently going to be improved through the standardisation of ETD formats and this will add to the university’s reputation and credibility.
      
      In conclusion, the development of an ETD automatic guideline verification tool presents an opportunity to enhance efficiency as well as promote consistency in the quality of ETDs while alleviating the challenges faced in the process of manually checking for compliance consequently reducing the workload for students, supervisors and examiners alike. Binmakhashen, G. M., & Mahmoud, S. A. (2019). Document Layout Analysis: A Comprehensive Survey. ACM Comput. Surv., 52(6), 1–36.Mishra, B. K., & Kumar, R. (2020). Natural Language Processing in Artificial Intelligence.Razı, S., Glendinning, I., & Foltýnek, T. (2019). Towards Consistency and Transparency in Academic Integrity. Peter Lang Gmbh, Internationaler Verlag Der Wissenschaften.
    \end{abstract_online}
    
