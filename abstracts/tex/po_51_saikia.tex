
    \begin{abstract_online}{Total Exploring AI-driven strategies for enhancing the visibility of E-Theses in Shodhganga Repository}{%
        \underline{Kamani Perera }}$^{1}$,{Vivek Mr. Ranjan,}}$^{1}$,{Kumar K Manoj,}}$^{1}$, { Surbhi }}$^{1}$}{%
        }{%
        $^1$ INFLIBNET }
      Exploring AI-driven strategies for enhancing the visibility of E-Theses in Shodhganga RepositoryVivek Ranjan¹, Nabajit Saikia², Manoj Kumar K³., Surbhi⁴Shodhganga Repository, managed by the INFLIBNET Centre, serves as a national digital repository for Indian electronic theses and dissertations (ETDs). It plays a crucial role in the academic ecosystem by providing a platform for researchers to access an extensive collection of scholarly works, thereby contributing to academic and research excellence. Despite its significance, Shodhganga faces several challenges in managing ETDs, including inconsistent metadata quality, manual entry errors, and a lack of comprehensive metadata for some records. With over 500,000 theses from various universities across India and continuously increasing annually covering diverse subjects such as science, technology, humanities, and social sciences, the repository requires robust solutions to streamline its management processes and enhance user experience. AI has involved to facilitate the technological solutions in various sectors which need human intelligence. There are many areas including the creation of ETDs as well as disseminations of the knowledge by picking up knowledge snippets to help the researchers. This poster proposes the efforts and integration of Artificial Intelligence (AI) in creation and managing ETDs to addressing the challenges and improve ETD management in Shodhganga.
      
      Objectives•Primary Goal: To provide a comprehensive flowchart and step-by-step implementation plan for enhancing ETD management in Shodhganga using AI, based on literature review.•Specific Objectives:oIllustrate AI-driven solutions for improving metadata quality and consistency.oOutline steps to enhance search and retrieval functionalities.oPresent a roadmap for increasing user engagement and satisfaction.The implementation plan for these AI-driven solutions is divided into three phases: Research and Development, Testing as prototype, and Full-Scale Deployment. During the Research and Development phase, a detailed literature review will identify best practices in AI-driven ETD management, and current management practices in Shodhganga will be analyzed to pinpoint areas for improvement. AI models for metadata creation, search enhancement, and recommendations will be developed and tested. The Pilot Testing phase will involve implementing these solutions on a subset of ETDs, collecting user feedback, and refining the algorithms based on initial results. Finally, in the Full-Scale Deployment phase, the AI solutions will be rolled out across the entire Shodhganga repository, with continuous monitoring and iterative improvements to ensure optimal performance and adaptation to new data.Expected OutcomeIntegrating AI into Shodhganga’s ETD management will streamline processes, reduce manual workload, and enhance efficiency. Improved search, personalized recommendations, and better metadata will enhance user experience and discoverability. Integration with external databases will broaden research access, making Shodhganga a more valuable resource for the academic community.Visual Elements for PosterFigure 1: Flowchart of AI-driven ETD Management Process.Figure 2: Automated Metadata Extraction FlowchartFigure 3: Flowchart of Graph Machine Learning for Metadata Quality EnhancementFigure 4: Diagram Illustrating Semantic Search ImplementationFigure 5: Diagram Depicting Personalized Recommendation SystemFigure 6: Diagram Automated Content Tagging and ClassificationFigure 7: Metadata Standardization Process DiagramFigure 8: Integration External Databases with Diagram IllustrationFigure 9: Graph showing the expected outcome for the enhancement of ShodhgangaReferencesShodhganga, INFLIBNET Centre. Available at: [https://shodhganga.inflibnet.ac.in/]
    \end{abstract_online}
    
