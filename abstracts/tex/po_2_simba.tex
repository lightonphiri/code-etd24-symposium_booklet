
    \begin{abstract_online}{Assessing the Efficiency of Data Science Programs in Enhancing Big Data Analysis Skills among Health Libraries and Information Scientists.}{%
        \underline{ Rajabu Simba}$^{1}$} Haruna Hussein }$^{1}$}{%
        }{%
        $^1$ Ministry Of Health *\newline{}
            }
	
	Introduction:The healthcare sector has seen a significant shift in data analysis methodologies due to the rapid advancements in technology. Data Science programs have emerged as a promising way to enhance the capabilities of professionals in health libraries and Information Science fields in managing and analyzing large datasets, known as Big Data. However, there is a research gap in assessing the effectiveness of these programs in equipping Health Libraries and Information Scientists with the necessary skills for proficient Big Data analysis. This study aims to bridge this gap by evaluating the efficacy of Data Science programs tailored for Health Libraries and Information Science professionals. The research aims to provide insights into the impact of these educational initiatives on the skill development and proficiency of Health Libraries and Information Scientists in Big Data analysis, contributing to the advancement of knowledge in this evolving field.
	
	Aim:To assess and evaluate the effectiveness of Data Science programs in improving the capacity and proficiency of professionals working in health libraries and Information Science. The study seeks to ascertain whether these programs adequately equip participants with the necessary skills and knowledge required for proficiently handling and analyzing Big Data in healthcare contexts.
	
	Method:The study used a quantitative approach to gather data from 150 participants in health libraries and Information Science fields who completed Data Science programs in Tanzania. The survey evaluated their perceptions, skill sets, and confidence in executing Big Data analysis post-program completion. Statistical tools like descriptive statistics were used to analyze the collected data, providing insights into the effectiveness of Data Science programs in enhancing Big Data analysis skills.Results:The study aimed to assess the efficacy of Data Science programs through quantitative analysis of participant responses. Among the surveyed professionals who completed the program (n=150), a significant 82% reported a noticeable enhancement in their skill sets related to Big Data analysis. Moreover, 75% of participants expressed increased confidence in applying advanced analytical techniques acquired from the program. Notably, 90% of respondents indicated satisfaction with the program's content and structure, highlighting its effectiveness in imparting practical skills for handling Big Data in healthcare contexts. The statistical data underscores substantial improvements in skill acquisition and positive perceptions among professional’s post-program completion, affirming the efficacy of Data Science programs in enhancing Big Data analysis proficiency within health libraries and Information Science fields.
	
	Conclusion:The survey analysis shows that Data Science programs significantly improve professional skills in Big Data analysis, enhancing competence and confidence. These findings have implications for health libraries and Information Science, fostering a more agile workforce capable of utilizing advanced data analysis techniques for informed healthcare decision-making. However, limitations like survey bias and study scope need to be addressed for further improvement.  
    \end{abstract_online}
    
