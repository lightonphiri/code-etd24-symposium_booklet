
        \begin{abstract}Total Quality Management in ETD Repository at Chartered Institute Management (CIPM)}{%
          Kamani Perera }{%
          Chartered Institute of Personnel Management }{%
            }
      Introduction: As the landscape of education and professional development evolves, the use of Electronic Theses and Dissertations (ETDs) has become integral to knowledge dissemination and research advancement. The Chartered Institute of Personnel Management (CIPM) recognizes the significance of its ETD repository as a valuable resource for HRM professionals.
      This study focuses on implementing Total Quality Management (TQM) principles in the ETD repository at CIPM to ensure efficiency, accessibility, and the highest standards of content.
      
      Objectives: 
      1.To identify the areas for improvement through TQM principles
      2.To assess the impact of these enhancements on user experience and knowledge dissemination
      3.To make recommendations to improve the quality of ETD repositories in order to provide effective service to the CIPM stakeholders
      
      Methodology: A comprehensive approach was employed, starting with an analysis of the existing ETD repository at CIPM to identify strengths and areas needing improvement. TQM principles were then applied, including process mapping, stakeholder engagement, and continuous improvement strategies. A survey was conducted among HRM professionals and researchers utilizing the ETD repository to gather feedback on usability, content relevance, and accessibility. Additionally, interviews were conducted with repository administrators and stakeholders to gain insights into challenges and opportunities for enhancement.Results: The assessment of the ETD repository at CIPM revealed strengths in terms of content variety and relevance to HRM professionals. However, areas for improvement were identified, including navigation complexity, search functionality, and metadata consistency. Through the application of TQM principles, enhancements were implemented, such as improved metadata tagging, streamlined search features, and user-friendly interface updates. The survey results indicated a positive reception to these enhancements, with users reporting increased satisfaction with the repository's usability and content accessibility.
      
      Conclusions: In conclusion, this study highlights the importance of implementing TQM principles in the ETD repository at CIPM to ensure it remains a valuable and user-centric resource for HRM professionals. The enhancements made through this process have resulted in a more efficient and user-friendly repository, aligning with the institute's commitment to excellence in knowledge dissemination. Recommendations include regular audits of the repository, ongoing stakeholder engagement, and continuous improvement initiatives to uphold TQM standards. By embracing TQM in the ETD repository, CIPM can continue to support the research and professional development needs of HRM professionals, contributing to the advancement of the field. This research serves as a model for other organizations seeking to optimize their ETD repositories through TQM principles, ultimately fostering a culture of quality and innovation in knowledge management within the HRM community.

        \end{abstract}
        
