
    \begin{abstract_online}{Empowering HRM Professionals: Advancing Research Culture with ETDs in The Chartered Institute of Personnel Management (CIPM), Sri Lanka}{%
        \underline{Kamani Perera}$^{1}$, Anushka EarskinK $^{1}$, Indika Wijayasriwardana $^{1}$, Heather Fernando)$^{1}$}{%
        }{%
        $^1$ Chartered Institute Of Personnel Management, Sri lanka}
            }
      Introduction: The Chartered Institute of Personnel Management (CIPM) plays a key role in the professional development of Human Resource Management (HRM) practitioners. To enhance this development, there is a growing recognition of the importance of fostering a culture of advanced research within the HRM community. This study focuses on the utilization of Electronic Theses and Dissertations (ETDs) as a means to empower HRM professionals in CIPM, enabling them to engage in rigorous research practices.Objectives: •To assess the current state of research culture among HRM professionals in CIPM•To identify the potential benefits of ETDs in advancing the research culture, and •To provide recommendations for integrating ETDs effectively into the professional development framework in the field of HRM
      
      Methodology: A mixed-methods approach was employed, starting with a comprehensive literature review to establish the foundation of research culture, the significance of ETDs, and their potential impact on HRM practitioners. This was followed by a survey distributed among a sample of HRM professionals within CIPM to gather insights into their awareness, usage, and perceptions of ETDs.Additionally, qualitative interviews were conducted with HRM leaders to delve deeper into the challenges and opportunities associated with integrating ETDs into professional development.
      
      Results: The findings reveal a varied kind of utilization of ETDs among HRM professionals in CIPM. While there is a general awareness of ETDs, their full potential in enhancing research skills and knowledge dissemination is yet to be realized. Challenges such as limited access to ETD repositories and inadequate training opportunities were identified. However, respondents expressed enthusiasm for incorporating ETDs into their professional development, higher education activities recognizing the value they hold in expanding their research capabilities and staying updated with current HRM trends.Conclusions: In conclusion, this study highlights the importance of advancing research culture among HRM professionals in CIPM through the effective utilization of ETDs. Recommendations include developing tailored training programs on ETD usage, improving access to ETD repositories, and nurturing collaboration between HRM professionals to share research findings. By embracing ETDs, HRM professionals can not only enhance their individual skills but also contribute to the overall knowledge base within the field. This research serves as a call to action for CIPM and similar organizations to prioritize the integration of ETDs into their professional development strategies, ultimately empowering HRM professionals to excel in their roles and drive innovation in the field for the sustainable economy of the country.
    \end{abstract_online}
    
