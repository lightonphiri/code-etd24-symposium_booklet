\begin{abstract}{Enhancing Electronic Thesis and Dissertation Visibility: A Focus on Institutional Repository Platforms and Inherent Challenges in the Nigerian Context}{%
           David, Okhakhu O }{%
            Lead City University, Ibadan }{%
            }
       Electronic Theses and Dissertations (ETDs) are an essential part of academic work since they provide insightful analysis and significant contributions to the international academic community. Platforms for institutional repositories (IRs) are vital conduits for the international distribution and availability of ETDs. However, there is a notable disparity in the visibility and accessibility of ETDs due to particular difficulties with the implementation and efficacy of IR platforms in Nigerian institutions. With a focus on the difficulties Nigerian institutions confront, this paper examines how institutional repository platforms might improve the visibility of ETDs globally. The study emphasises the transformative potential of IR platforms in boosting the impact and dissemination of ETDs through a comprehensive evaluation of the literature. It also emphasises how critical it is to fill up the knowledge vacuum in the literature about the application and efficacy of IR platforms in Nigerian academic institutions.This study used a mixed-methods approach, combining qualitative investigation through stakeholder surveys and interviews with Nigerian institutions' stakeholders with quantitative analysis of IR availability and usage data. The quantitative analysis looks at IR availability, citation counts, download trends, and ETD submission rates on IR platforms to get a sense of how visible Nigerian ETDs are right now. Qualitative approaches explore the particular obstacles Nigerian institutions have in efficiently leveraging IR platforms, such as infrastructure limits, institutional support, cultural considerations, and knowledge gaps. Compiling quantitative and qualitative data allows for a more thorough understanding of the intricate relationships surrounding ETD deployment and visibility in Nigerian context. The study exposes inherent challenges to the efficient use of IR platforms. These results demonstrate how urgently customised interventions are needed to solve these issues and raise Nigerian ETDs' profile internationally. The study makes further recommendations based on its findings on the use of IR platforms in Nigerian institutions. These recommendations include making investments in technology infrastructure, starting capacity-building programmes, proposing a cultural mind shift, advocating for open access laws in policy, and encouraging stakeholder collaboration. Through implementation of these suggestions, Nigerian establishments can surmount current obstacles and utilise IR channels to augment the prominence and influence of their ETDs on the international academic sceneThis study clarifies the unique difficulties encountered by Nigerian institutions while also enhancing knowledge of the function of institutional repository platforms in increasing ETD visibility. Through the implementation of focused interventions and the resolution of these issues, stakeholders can strive to promote increased inclusivity and involvement in the worldwide ecosystem of academic communication.Keywords: Nigerian Electronic Theses and Dissertations (ETDs), Institutional Repository Challenges, Global Research Visibility Gap, Open Access Scholarship, Digital Divide Impact, Nigeria Universities.

        \end{abstract}
