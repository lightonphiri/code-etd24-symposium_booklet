
    \begin{abstract_online}{The Conversion of Printed Theses and Dissertations into Digital Formats: A Case Study of the University of South Africa Library}{%
    \underline{Mduduzi Ntetha}$^{1}$}{%
    }{%
    $^1$ UNISA *\newline{}
}

Printed theses and dissertations in academic libraries are primarily accessible only to patrons of that particular library. Converting these printed copies into digital formats can mitigate the risk of losing valuable materials due to age and prolonged shelf time, while also enhancing the visibility of the university's research output. This study explores the process of digitizing printed materials in the UNISA Institutional Repository (UnisaIR), an open-access digital platform that hosts the research and intellectual output of the UNISA community, including electronic theses and dissertations (ETDs). The UnisaIR is globally accessible via various internet browsers and search engines, thereby promoting UNISA’s research output and increasing the visibility of African scholarship.

The study employs observation as the primary data-gathering technique, focusing on the digitization process. This approach facilitates a thorough and nuanced understanding of the digitization phenomenon as it occurs in real time. Key findings indicate that converting printed theses and dissertations into digital formats not only helps prevent loss due to aging but also supports the objectives of Sustainable Development Goal 4 by providing easier access to these materials for the international community.

In conclusion, this original study offers valuable insights into the digitization process within the UnisaIR, highlighting its potential benefits for both preservation and accessibility. Specifically, the findings demonstrate that digitizing printed theses and dissertations plays a critical role in safeguarding these materials and enhancing the visibility of the university's research output and African scholarship. Based on the study's findings, it is recommended that academic libraries implement digitization programs for their printed collections, particularly for materials at risk of deterioration. This initiative can help ensure long-term preservation while making these resources more accessible to a broader audience.

Keywords: Institutional Repository, ETD, open access, academic library
\end{abstract_online}

