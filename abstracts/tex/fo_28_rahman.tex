
    \begin{abstract_online}{Determining the Factors Influencing the Utilization of Open Source Digital Repository Software in the Preservation of ETDs in Academic Libraries in Bangladesh}{%
    \underline{Dr. Md. Zillur Rahman}$^{1}$}{%
    }{%
    $^1$ Ahsanullah University of Science and Technology, Bangladesh
}

Problem/Motivation/Goal: This article aims to identify the factors affecting the utilization of open source digital repository software (DRS) for the preservation of Electronic Theses and Dissertations (ETDs) in academic libraries across Bangladesh. The study also examines compliance challenges with standards and protocols such as metadata, WebDAV, OpenSearch, OpenURL, RSS, ATOM, the Open Archives Initiative (OAI)-PMH, OAI-ORE, SWORD, and WebDAV for access, ingest, and export when using open source DRS. Furthermore, the relationship between the technical and financial aspects of employing DRS is explored.

**Methodology/Approach**: This investigation will employ a quantitative methodology. A structured questionnaire will be distributed to academic librarians in Bangladesh to meet the study objectives. The questionnaire will be shared through various forums, email groups, and mailing lists related to libraries. It will consist of three sections (A, B, and C). Section A will gather respondents' demographic information, Section B will focus on factors related to employing DRS for the preservation of ETDs, and Section C will address issues and complaints regarding the use of DRS. A 7-point Likert scale will be utilized to gauge the degree of responses. Descriptive statistics will be employed to analyze the collected data, alongside statistical significance tests to evaluate the relationships between the technical and economic factors influencing DRS usage.

Objectives: The primary objective of this study is to explore the factors influencing the use of open source DRS. The specific objectives are:
1. To determine the factors affecting the use of DRS in the preservation of ETDs.
2. To identify the complaints regarding the use of DRS for the preservation of ETDs.

Research Questions
1. What is the level of factors influencing the usage of open source DRS by academic libraries in Bangladesh?
2. What is the relationship between multiple factors and the usage of open source DRS?
3. What complaint issues are reported by academic institutions in Bangladesh?

Anticipated Results: Despite the advantages of open source DRS—such as the ability to capture and ingest ETDs along with associated metadata, providing easy access to ETDs through listing and searching, and ensuring long-term preservation—there are notable drawbacks. These include a flat file and metadata structure, poor user interface, lack of scalability and extensibility, limited API, restricted metadata features, inadequate reporting capabilities, and insufficient support for linked data. Overcoming these barriers, the utilization of open source DRS has been increasing steadily in Bangladesh. Most open source DRS are functional and based on the OAI-PMH. DSpace remains the preferred software for DRS content management. However, many DRS lack content control policies and do not provide usage data, with English being the predominant language for materials. A considerable number of institutional repositories (IRs) incorporate Web 2.0 tools, with RSS being the most utilized. Many IRs have not modified their user interfaces, and the majority maintain a bilingual interface.

According to Crow (2002) and Rahman (2015), DRS in Bangladesh essentially began as a part of the institution's digital repository, defined as “[...] a digital archive of the intellectual products created by the faculty, research staff, and students of an institution, accessible to end-users both within and outside the institution, with few if any barriers to access.” Academic libraries in Bangladesh have chosen open source software for their institutional repositories and comprehensive digital library platforms. In the last twenty years, the computerization and digitization of libraries in Bangladesh have gained significant momentum, marking an impressive achievement (Rahman et al., 2015).

\end{abstract_online}

