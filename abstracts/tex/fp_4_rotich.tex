
        \begin{abstract}{The changing landscape in research data management in Kenya’s Universities: an analysis of development and implementation  {%
            Rotich, Kenneth K }{%
            Egerton university, Kenya}{%
            }
  There is move towards Open Science with the goal of making research data accessible and reusable by the research community is a growing global trend. Research data is key pillar to Open Science. Research institutions are generating research data in huge magnitude which requires well planned and though out strategies for ensuring that they are available when needed. This has resulted in implementation of research data management (RDM) strategies as critical component for any research organization for enhancement of quality research output. Development of policies and guidelines by Universities in Kenya is critical in implementation and provision of in Research data management services.Institutions in developed countries have already developed policies which require researchers to deposit their research data in open repositories. With the globalization this trend is finding its way to developing countries such as kenya as evident in the development of research data management services in countries such as South Africa. In Kenya, there are few indications that point to existence of research data management services such as awareness training. There is evidently huge research activities in research institutions in Kenya, however management of research data that are acquired during the research processes seems to be largely ignored or given inadequate attention that it deserves. The impact of RDM on innovations has consequently not been documented. This begs the questions as to what extend are research data is managed in academic institutions in Kenya. These study will therefore unravel the state of development and implementation of RDM services in academic Libraries in Kenya. The analysis draws on a review of existing RDM policies at Kenyan universities, interviews with key stakeholders, and a survey of researcher awareness and practices. The findings will inform the development of a framework for robust RDM policy design and implementation in the Kenyan university context. This framework will contribute to improved research data management practices, fostering transparency, reproducibility, and ultimately, the impact of Kenyan research.
        \end{abstract}
        
