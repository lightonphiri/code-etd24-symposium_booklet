\begin{abstract}{Design and Implementation of an Interoperable Zambia National Electronic Thesis and Dissertation Portal}{%
        Elijah Chileshe}{%
            The University of Zambia, Zambia}{%
            }

        Introduction: Zambia boasts a significant number of Higher Learning Institutions (HEI) contributing to a vibrant academic landscape marked by prolific scholarly research output. The Higher Education Authority (HEA) reports a total of 123 HEIs as of 2024 (Higher Education Authority, 2018). A number of the 123 HEIs offer postgraduate programmes which, in part, will require students to produce a thesis or dissertation manuscripts. Increasingly, such manuscripts are stored in electronic form, resulting in so-called Electronic Theses and Dissertations (ETDs). Existing literature reports that a total of 10 HEIs have implemented Institutional Repositories (IRs), with eight (8) of them still functional (Chisale \& Phiri, 2023). Motivation: With the increase in the number of HEIs offering postgraduate programmes and, additionally, the rise in the adoption and implementation of IRs, discoverability of such scholarly outputs is compromised. In addition, assessment of the quality of ETDs by entities such as HEA is challenging. A potential solution to this problem is the implementation of a platform to provide centralised access to the ETDs.

        Methodology: A national ETD portal has been implemented and primarily makes use of the Open Archives Initiative Protocol for Metadata Harvesting (OAI-PMH) to harvest ETD metadata from HEI IRs across Zambia. By leveraging OAI-PMH, the portal ensures efficient and standardized retrieval of metadata, enabling seamless integration of ETDs into its centralized repository. Additionally, ongoing monitoring and maintenance will be conducted to uphold the functionality and usability of the portal.

        Results: It is expected that the centralised repository of ETDs will significantly improve accessibility to scholarly resources for researchers in Zambia. By providing a single platform for accessing ETDs from various institutions, researchers will be able to save time and effort in searching for relevant materials. It is expected that the portal will lead to increased visibility and recognition of research output from higher learning institutions in Zambia. 

        Chisale, A., \& Phiri, L. (2023, November 17). Towards Metadata Completeness in National ETD Portals for Improved Discoverability. 26th International Symposium on Electronic Theses and Dissertations. ETD 2023, Gujarat, India. http://docs.ndltd.org/metadata/etd2023/9/index.html

        Higher Education Authority. (2018, January 4). Higher Education Authority. Higher Education Authority - Ensuring Quality in Higher Education; Higher Education Authority. https://hea.org.zm

\end{abstract}

