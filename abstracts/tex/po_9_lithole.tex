\begin{abstract_online}{The University of Johannesburg’s Journey to Enhance the Content of the Institutional Repository (IR) and Improve Discoverability}{%
    \underline{Mutali M Lithole}$^{1}$}{%
    }{%
    $^1$ University of Johannesburg\newline{}
}

Institutional Repositories (IRs) have been well established in most academic libraries over the past few decades. As technology has developed across various domains, similar advancements have been observed in the software and tools used within the IR environment. Prominent systems such as DSpace, EPrints, and Fedora are widely recognized in this context (Castagne, 2013). The University of Johannesburg (UJ) library established its IR in 2008, which currently hosts a variety of research outputs generated by the university, including journal articles, conference proceedings, books, and book chapters.

Over the past 14 years, the library has embarked on a journey of exploration and experimentation with several systems to enhance the content of the IR and increase its visibility. These systems include DSpace, Vital, and most recently, Exlibris Esploro. The purpose of this paper is to share our experiences with these systems and the software utilized throughout this journey. 

The presentation will begin with a brief history of our IR development, followed by a technical discussion on the systems employed for storing and accessing content, as well as the methods used to collate relevant materials. The latter part of the presentation will focus on the latest system implemented at UJ, Exlibris Esploro, which enables us to maximize the impact of our institutional research through intelligent data capture. This innovation not only reduces the workload of IR staff but also enhances the visibility of both the content and its authors.

Keywords: Open access, institutional repositories, Esploro, University of Johannesburg  
Further reading: [https://www.usetda.org/resources/institutional-repository-software/](https://www.usetda.org/resources/institutional-repository-software/)
\end{abstract_online}

