
    \begin{abstract_online}{Future-Proofing Research by Long-term ETD Preservation: Challenges and Opportunities }{%
        \underline{Shahzeb Hasan}$^{1}$, Subhajit Panda$^{2}$}{%
        }{%
        $^1$ Akal University, India\newline{}$^2$ Chandigarh University}
            }
              This paper examines the challenges and opportunities in preserving Electronic Theses and Dissertations (ETDs). It highlights the significance of ETDs in academia and public knowledge while addressing technological hurdles such as digital obsolescence, data integrity, cybersecurity threats, and infrastructure limitations. The paper explores technological advancements like digital preservation tools, open-source platforms, cloud storage, and automation that can mitigate these challenges. Organizational and policy issues, including institutional policies, funding, legal considerations, and the need for collaboration, are also discussed. The study advocates for comprehensive preservation policies, strategic funding, enhanced legal frameworks, and strengthened institutional collaboration. It calls for a proactive approach to ensure ETDs' long-term accessibility and reliability, urging stakeholders to prioritize and invest in preservation efforts to safeguard these critical academic resources for future generations.

        \end{abstract}
    \end{abstract_online}
    
