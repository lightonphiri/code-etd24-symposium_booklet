\begin{abstract_online}{Landscape of Open Access Repositories with Special Reference to Electronic Theses and Dissertations (ETD) across SAARC and BRICS Nations: A Comparative Analysis}{%
    \underline{Vivek Ranjan}$^{1}$, Manoj Kumar K.$^{1}$, Kotrayya Agadi$^{2}$, Surbhi$^{1}$}{%
    }{%
    $^1$ INFLIBNET, India\newline{}$^2$ Central University of Gujarat, India}

Research outputs stored as softcopy and archived through open access are gaining momentum across the globe. Academic and research institutes, meticulously following research guidelines and policies, are systematically producing and storing research results in sophisticated ETDs in almost every nation. In the digital age, global visibility of research is crucial, with Electronic Theses and Dissertations (ETDs) playing a vital role. Open Access Repositories (OARs) have gained traction, led by Europe, North America, and increasingly by Asia. Key repositories like ProQuest Dissertations & Theses Global and DART-Europe are instrumental in this movement. Noteworthy ETD repositories in SAARC and BRICS nations include Shodhganga in India, maintained by the INFLIBNET Centre; the Digital Archive on Agricultural Theses and Journals in Bangladesh; the Pakistan Research Repository; and the Biblioteca Digital Brasileira de Teses e Dissertações in Brazil. Although India started its repository a bit late in 2010, it has gradually become a significant contributor to the ETD landscape. However, a few challenges remain in establishing comprehensive ETD repositories, given the diverse characteristics of these nations in terms of population size and higher education institutions. Key concerns include the structure and architecture of ETDs, workflows of submissions and authentication, metadata standards used, harvesting methods implemented, scalability, interoperability, and DRM issues. A detailed analysis of ETD repositories across SAARC and BRICS nations is essential to identify commonalities and differences.

Objectives:
1. To compare open access ETD repositories in SAARC and BRICS countries, focusing on subject coverage and regional language contributions.
2. To investigate the software used for open access ETD repositories across SAARC and BRICS nations.
3. To analyze copyright policies relevant to open access ETD repositories across SAARC and BRICS nations.
4. To evaluate persistent metadata standards within the open access ETD repositories across SAARC and BRICS nations.

The available open-access ETD repositories under Open ROAR/DOAR are examined to derive the sample size for analysis. The study employs a systematic approach for ETD data collection and analysis, identifying active ETD repositories based on accessibility and content relevance. Information on software platforms, copyright policies, metadata standards, and subject coverage is gathered. Both thematic analysis and quantitative methods are used to identify patterns and trends.

This analysis encompasses essential steps such as conducting thorough searches to identify active ETD repositories, gathering repository information on software platforms, copyright policies, and metadata standards, categorizing ETDs based on subject classifications and language assessment, developing a search strategy for AI-related ETDs, and performing both thematic and quantitative analyses to discern patterns and trends in the data.

The findings reveal a varied landscape of ETD repositories across SAARC and BRICS nations. India's Shodhganga is a significant contributor, reflecting the country's leadership in this area. Differences in repository management software and metadata standards are highlighted, with some countries adopting more advanced systems than others. Copyright policies also vary, affecting the accessibility and use of ETDs. (Details will be available in the full paper.)

The analysis underscores the diversity of open access ETD repositories across SAARC and BRICS nations, differing in subject coverage, language representation, software, copyright policies, metadata standards, and interoperability. By spotlighting regional languages and AI research, this study suggests a democratization of knowledge access and innovation.

\end{abstract_online}

