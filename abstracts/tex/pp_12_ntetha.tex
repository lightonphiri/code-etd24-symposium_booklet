
        \begin{abstract}{The Conversion of Printed Theses and Dissertations into Digital Formats: A Case Study of the University of South Africa Library}{%
    Mduduzi Ntetha}{%
    UNISA}{%
}

Printed theses and dissertations in academic libraries are mainly accessible by patrons of that particular library. Converting printed copies of theses and dissertations into digital format can help eliminate the problem of losing these materials due to their age and the length of time they have been on the shelves, and also make visibility of the university's research output a reality. This study explores the process of converting printed materials into digital format in the UNISA Institutional Repository (UNISAIR) as a subsidiary of an academic library—an open access digital source that hosts the research and intellectual output produced by members of the UNISA community, including electronic theses and dissertations (ETDs). The UNISAIR is globally accessible via various internet browsers and search engines, advancing UNISA’s research output and enhancing the visibility of African scholarship pertaining to ETDs.

The study employs observation as the main technique for gathering data, with an emphasis on the digitization process. This method enables a thorough and nuanced comprehension of the digitization phenomena as it occurs in real time. The key findings indicate that the process of converting printed copies of theses and dissertations into digital format can help eliminate the problem of losing these materials due to their age and the length of time they have been on the shelves. On the other hand, the process serves as a positive drive to attain the targets of Sustainable Development Goal 4 by providing ease of use for the international community who may want to access these materials.

In conclusion, the originality of the study provides valuable insights into the process of converting printed materials into digital format in the UNISAIR, highlighting the potential benefits of this process for both preservation and accessibility. Specifically, the study has shown that digitizing printed copies of theses and dissertations plays a vital role in eliminating the risk of losing these materials due to age and shelf time while also enhancing the visibility of the university's research output and African scholarship. Based on the findings of the study, it is recommended that academic libraries consider implementing digitization programs for their printed collections, particularly for materials that are at risk of being lost due to age or other factors. This can help ensure the long-term preservation of these materials while also making them more accessible to a wider audience.

Keywords: Institutional Repository, ETD, open access, academic library

\end{abstract}

