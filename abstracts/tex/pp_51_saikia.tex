\begin{abstract}{Exploring AI-Driven Strategies for Enhancing the Visibility of E-Theses in Shodhganga Repository}{%
    Nabajit Saikia}{%
    INFLIBNET}{%
}

Shodhganga Repository, managed by the INFLIBNET Centre, serves as a national digital repository for Indian electronic theses and dissertations (ETDs). It plays a crucial role in the academic ecosystem by providing a platform for researchers to access an extensive collection of scholarly works, thereby contributing to academic and research excellence. Despite its significance, Shodhganga faces several challenges in managing ETDs, including inconsistent metadata quality, manual entry errors, and a lack of comprehensive metadata for some records. With over 500,000 theses from various universities across India, continuously increasing annually and covering diverse subjects such as science, technology, humanities, and social sciences, the repository requires robust solutions to streamline its management processes and enhance user experience.

Artificial Intelligence (AI) has emerged as a transformative technology in various sectors, facilitating solutions that require human-like intelligence. This poster proposes the integration of AI in the creation and management of ETDs to address these challenges and improve ETD management in Shodhganga.

Objectives:
- Primary Goal:To provide a comprehensive flowchart and step-by-step implementation plan for enhancing ETD management in Shodhganga using AI, based on a literature review.
Specific Objectives:
  1. Illustrate AI-driven solutions for improving metadata quality and consistency.
  2. Outline steps to enhance search and retrieval functionalities.
  3. Present a roadmap for increasing user engagement and satisfaction.

The implementation plan for these AI-driven solutions is divided into three phases:
1. Research and Development: A detailed literature review will identify best practices in AI-driven ETD management. Current management practices in Shodhganga will be analyzed to pinpoint areas for improvement. AI models for metadata creation, search enhancement, and recommendations will be developed and tested.
2. Pilot Testing:This phase will involve implementing these solutions on a subset of ETDs, collecting user feedback, and refining the algorithms based on initial results.
3. Full-Scale Deployment: The AI solutions will be rolled out across the entire Shodhganga repository, with continuous monitoring and iterative improvements to ensure optimal performance and adaptation to new data.

Expected Outcomes:
1. Integrating AI into Shodhganga’s ETD management will streamline processes, reduce manual workload, and enhance efficiency. Improved search, personalized recommendations, and better metadata will enhance user experience and discoverability.
2. Integration with external databases will broaden research access, making Shodhganga a more valuable resource for the academic community.

Visual Elements for Poster:
- Figure 1: Flowchart of AI-driven ETD Management Process.
- Figure 2: Automated Metadata Extraction Flowchart.
- Figure 3: Flowchart of Graph Machine Learning for Metadata Quality Enhancement.
- Figure 4: Diagram Illustrating Semantic Search Implementation.
- Figure 5: Diagram Depicting Personalized Recommendation System.
- Figure 6: Diagram of Automated Content Tagging and Classification.
- Figure 7: Metadata Standardization Process Diagram.
- Figure 8: Integration of External Databases with Diagram Illustration.
- Figure 9: Graph Showing the Expected Outcome for the Enhancement of Shodhganga.

References:Shodhganga, INFLIBNET Centre. Available at: [https://shodhganga.inflibnet.ac.in/]

\end{abstract}

