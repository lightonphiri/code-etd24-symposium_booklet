\begin{abstract}{Automatic Electronic Thesis and Dissertation Guideline Verification For Consistently Formatted Manuscripts}{%
    Mubanga C. Chibesa}{%
    }{%
    The University Of Zambia}
    
Introduction: Higher Education Institutions worldwide enforce guidelines and academic approaches to ensure scholarly integrity and adherence to academic standards (Razı et al., 2019). The University of Zambia is not an exception. Like most HEIs, it offers training to postgraduate students, and one of the key aspects of postgraduate training is producing an Electronic Thesis and Dissertation manuscript. The Directorate of Research and Graduate Studies (DRGS) at the University of Zambia provides guidelines that stipulate how ETDs should be formatted. However, the process of checking for conformance is a manual and tedious procedure, resulting in the submission of inconsistently formatted manuscripts in the Institutional Repository (IR). To address this challenge, our project seeks to implement a tool that will automate the process of checking ETD compliance against established postgraduate guidelines. The tool will leverage data mining techniques to perform these tasks. More specifically, Document Layout Analysis (DLA) (Binmakhashen and Mahmoud, 2019) will be the core approach used in the implementation. The tool will flag portions of ETD manuscripts that do not conform to established guidelines. Hence, this will help resolve the inconsistencies in the format of submitted manuscripts.

Methodology: Using a mixed methods approach, document analysis will be employed to understand the postgraduate guidelines stipulated in the “Regulations and Guidelines for Postgraduate Studies” document; content analysis will be used on randomly sampled ETDs to experimentally determine the extent of the problem. Finally, a DLA Natural Language Processing model will be developed and evaluated using standard DLA metrics such as Structure Similarity Index and Intersection over Union.

Results: This study is part of ongoing work aimed at developing an automated tool that will verify ETD manuscripts' compliance with postgraduate guidelines. Upon successful completion of this project, we anticipate a reduced workload associated with the manual checking of manuscript conformity to postgraduate guidelines, freeing up time and resources to handle other academic tasks like focusing more on the content of the manuscripts. Not only that, academic integrity will consequently improve through the standardization of ETD formats, adding to the university’s reputation and credibility.

In conclusion, the development of an ETD automatic guideline verification tool presents an opportunity to enhance efficiency and promote consistency in the quality of ETDs while alleviating the challenges faced in the manual checking process, consequently reducing the workload for students, supervisors, and examiners alike. 

References:
- Binmakhashen, G. M., and Mahmoud, S. A. (2019). Document Layout Analysis: A Comprehensive Survey. ACM Comput. Surv., 52(6), 1–36.
- Mishra, B. K., and Kumar, R. (2020). Natural Language Processing in Artificial Intelligence.
- Razı, S., Glendinning, I., and Foltýnek, T. (2019). Towards Consistency and Transparency in Academic Integrity. Peter Lang Gmbh, Internationaler Verlag Der Wissenschaften.

\end{abstract}

