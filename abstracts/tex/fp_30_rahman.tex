
        \begin{abstract}{ ETDS in Ensuring Quality Education for Economic Growth to Achieve Sustainable Development Goals (SDGs):  Experience of SAARC Countries}{% MDr. Md. Zillur Rahman }{%
              Ahsanullah University of Science and Technology* {%
            }
  The purpose of this paper is to illustrate trend of economic growth, supply of skilled manpower, to find a correlation between them and role of ETDs in producing human capital who will meet the demand of SDGs among the SAARC Countries: Afghanistan, Bangladesh, Bhutan, India, Maldives, Nepal, Pakistan, and Sri Lanka. The paper describes also ETD as an element of research and higher studies, and its importance, nature, platform and initiatives in innovation to achieve SDGs. 
	
	The main objectives of this paper is to find : 
	1.New role of libraries due to Agenda 2030 of UN around the world 
	2.Contribution of Open content provided by UN Research4Life initiatives;
	3.Role of ETDs and NDLTDs around the world;d.Accelerating economic growth in SAARC countries ensuring quality education and skilled manpower through using ETDs and NDLTDs
	4.Recommends to overcome all barriers in implementation these two goals.Access to information was recognized as a priority in SDG 16 (target 16.10): ensuring public access to information and safeguarding fundamental freedoms in compliance with national legislation and international agreements. Culture (target 11.4), climate literacy (SDG-13) and ICT (target 5b, 9c, 17.8) have also been included in the SDGs. And universal literacy is recognized in the vision for the UN 2030 Agenda. Obtaining a quality education (QE) is the foundation to creating sustainable development (SD). In addition to improving quality of life, access to inclusive education can help equip locals with the tools required to develop innovative solutions to the world’s greatest problems (United Nations, 2015). Without having a QE no economic growth is possible. To provide QE, ETDs have a great role. ETDs provide access to new scholarly knowledge, scientific publications and valuable information to accelerate innovation in SD. This research expects to reveal the impacts of ETDs in QE, innovation in SD in SAARC countries. Describes the SDGs and initiatives for creating a livable world for all. 
	
	Methodology:The paper is qualitative in nature. In order to highlight the importance of ETDs as a part of quality education in SAARC countries, data were collected from the following sources: United Nations, World Bank Databank, World Development Indicators, World Economic Forum, International Monetary Fund (IMF) and similar organizations. In addition, national portals of the concerned countries, current literature, research and personal initiatives have been included as well.Expected resultsExisting research in Library and Information Science is relevant to individual SDGs serves as a link between them. Similarly SDGs focus new sites for empirical as well as inviting innovation in Library and Information Science’s contribution to SD debates. ETDs are importance tools for scholarly communication (SC) and innovation. ETDs are becoming one of the vital elements and have been creating significant impact on the creation of new knowledge, dissemination and prevent repetition in research arena. The SAARC countries are growing economies. However, education and human development are slower in comparison to developed nations. Rate of new technology adaptation are also comparatively slower than developed countries. 
	
	“The results and recommendations contained in the paper should be of interest to authors, policy makers, funding agencies and information professionals in both developing and developed countries (Chan & Costa, 2005).”ConclusionSD depends on the success of 17 SDGs which are dependent on each other. Success on one goal individually brings no achievement in overall mankind. To this end ensuring quality education to sustainable economic growth is utmost important and role of ETDs is inevitable. 
      

        \end{abstract}
        
