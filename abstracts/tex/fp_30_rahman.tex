
        \begin{abstract}{ETDS in Ensuring Quality Education for Economic Growth to Achieve Sustainable Development Goals (SDGs): Experience of SAARC Countries}{%
    MDr. Md. Zillur Rahman}{%
    Ahsanullah University of Science and Technology*}

The purpose of this paper is to illustrate the trend of economic growth and the supply of skilled manpower, to find a correlation between them and the role of ETDs in producing human capital who will meet the demand of SDGs among the SAARC countries: Afghanistan, Bangladesh, Bhutan, India, Maldives, Nepal, Pakistan, and Sri Lanka. The paper also describes ETDs as an element of research and higher studies, and their importance, nature, platform, and initiatives in innovation to achieve SDGs.

The main objectives of this paper are to find:
1. New role of libraries due to Agenda 2030 of the UN around the world.
2. Contribution of open content provided by UN Research4Life initiatives.
3. Role of ETDs and NDLTDs around the world.
4. Accelerating economic growth in SAARC countries, ensuring quality education and skilled manpower through the use of ETDs and NDLTDs.

Access to information was recognized as a priority in SDG 16 (target 16.10): ensuring public access to information and safeguarding fundamental freedoms in compliance with national legislation and international agreements. Culture (target 11.4), climate literacy (SDG-13), and ICT (target 5b, 9c, 17.8) have also been included in the SDGs. Universal literacy is recognized in the vision for the UN 2030 Agenda. Obtaining quality education (QE) is the foundation for creating sustainable development (SD). In addition to improving quality of life, access to inclusive education can help equip locals with the tools required to develop innovative solutions to the world’s greatest problems (United Nations, 2015). Without QE, no economic growth is possible. To provide QE, ETDs play a significant role. ETDs provide access to new scholarly knowledge, scientific publications, and valuable information to accelerate innovation in SD. This research expects to reveal the impacts of ETDs on QE and innovation in SD in SAARC countries and describes the SDGs and initiatives for creating a livable world for all.

Methodology: The paper is qualitative in nature. To highlight the importance of ETDs as a part of quality education in SAARC countries, data were collected from the following sources: United Nations, World Bank Databank, World Development Indicators, World Economic Forum, International Monetary Fund (IMF), and similar organizations. In addition, national portals of the concerned countries, current literature, research, and personal initiatives have been included.

Expected Results: Existing research in Library and Information Science relevant to individual SDGs serves as a link between them. Similarly, SDGs focus on new sites for empirical as well as inviting innovation in Library and Information Science’s contribution to SD debates. ETDs are important tools for scholarly communication (SC) and innovation. They are becoming vital elements and have been creating significant impacts on the creation of new knowledge, dissemination, and prevention of repetition in the research arena. The SAARC countries are growing economies. However, education and human development are slower in comparison to developed nations. The rate of new technology adaptation is also comparatively slower than in developed countries.

“The results and recommendations contained in the paper should be of interest to authors, policymakers, funding agencies, and information professionals in both developing and developed countries (Chan & Costa, 2005).”

Conclusion: Sustainable development depends on the success of the 17 SDGs, which are interdependent. Success in one goal individually brings no achievement for overall mankind. To this end, ensuring quality education for sustainable economic growth is of utmost importance, and the role of ETDs is inevitable.

\end{abstract}

