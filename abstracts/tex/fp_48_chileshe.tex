
        \begin{abstract}{A Pre-Processing Pipeline for Improved ETD Metadata Quality in Downstream Services } {% Elijah Chileshe}{%
            The University of Zambia, Zambia}{%
            }
       
       Introduction:
       	
       	The Zambia National Electronic Theses and Dissertations (ZNETD) Portal serves as a crucial platform for accessing Electronic Theses and Dissertations (ETDs) from Higher Education Institutions (HEIs) in Zambia. The ZNETD functions by automatically harvesting ETD metadata from HEI Institutional Repositories (IRs) across Zambia. However, its usefulness is hindered by the inconsistently formatted and poor quality of metadata. This paper outlines the core challenges surrounding metadata quality in the ZNETD Portal and proposes the implementation of a pre-processing pipeline to address metadata quality issues.
       
	Motivation:
       	
	There are a reported 10 IRs, out of which eight (8) of them are presently functional (Chisale & Phiri, 2023). The HEIs with functional repositories generally have different ingestions policies and practices, resulting in inconsistently formatted metadata. More importantly, existing studies have revealed that some HEIs lack IR policies (Kasonde & Phiri, 2023).Methodology:
	
	The work presented in this paper involved two (2) primary objectives:* Empirical evaluation of inconsistencies with ETD metadata from HEI IRs—Metadata from HEIs with functional IRs was harvested using the Open Archives Initiative Protocol for Metadata Harvesting (Open Archives Initiative Protocol for Metadata Harvesting, n.d.). The metadata elements were then analysed related to the ETD-ms metadata standard.* Design and implication of an ETD metadata preprocessing pipeline—Pre-processing pipeline scripts were implemented to format metadata consistently and, additionally, to address the issue of incomplete metadata.
	
	Anticipated Results:
	
	It is anticipated that the implementation of the pre-processing pipeline software tool will lead to a significant improvement in metadata quality within the Zambia National ETD Portal. Enhanced metadata consistency and completeness will facilitate more efficient resource discovery and retrieval, thereby enriching the user experience and maximising the utility of the platform.Chisale, A., & Phiri, L. (2023, November 17). Towards Metadata Completeness in National ETD Portals for Improved Discoverability. http://ir.inflibnet.ac.in/handle/1944/2446Kasonde, C. C., & Phiri, L. (2023). Assessing and Promoting Metadata Quality for Electronic Theses and Dissertations in Institutional Repositories Using a Policy-Driven Approach. INFLIBNET Centre, Gandhinagar.Open Archives Initiative Protocol for Metadata Harvesting. (n.d.). Retrieved May 18, 2024, from https://www.openarchives.org/pmh
        \end{abstract}
        
