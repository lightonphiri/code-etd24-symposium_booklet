\begin{abstract}{Characterization of Scientific Production on Electronic Theses and Dissertations Based on a Bibliometric Analysis}{%
    Juliana Sousa}{%
    Instituto Brasileiro de Informação em Ciência e Tecnologia, Brazil}{%
}

The text presents the importance of expanding the dissemination of electronic theses and dissertations (ETDs). Given this importance, the study proposes a bibliometric analysis of scientific production that focuses on ETDs. The time frame used for the analysis spans from 2002 to 2023. To assist in data retrieval, the Publish or Perish tool was employed. Consequently, 601 documents were identified, of which 61 were selected for detailed analysis. The results indicate a higher maturity in scientific publications in English, which constitute the majority of these publications. Given these findings, it is possible to infer that there are still gaps in the topic to be addressed, which could lead to greater incentives for further studies in this area.

\end{abstract}


