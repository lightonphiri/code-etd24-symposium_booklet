
        \begin{abstract}{IInternational Visibility of ETDs in Portuguese and in English on a Brazilian Repository}{%
            Anna B Pavani}{%
            Pontifícia Universidade Católica do Rio de Janeiro, Brazil}{%
            }
         01.Background and Introduction:Portuguese is the 9th most spoken language (4th Western language) in the world according to Berlitz (https://www.berlitz.com/blog/most-spoken-languages-world) or the 8th (4th Western according to Statista (https://www.statista.com/statistics/266808/the-most-spoken-languages-worldwide/). Both numbers mean that Portuguese has a large number of speakers. The Comunidade de Países de Língua Portuguesa (https://www.cplp.org/) lists nine countries that have Portuguese as one of the official languages – Angola (36.7M), Brazil (216.4M), Cape Verde (0.6M), Guinea-Bissau (2.2M), Equatorial Guinea (1.7M), Mozambique (33.9M), Portugual (10.2M), Sao Tome & Principe (0.2M) and Timor-Leste (1.4M). Their populations in 2023 were informed on WorloMeter (https://www.worldometers.info/world-population/population-by-country/). It can easely be observed that Brazil holds 71.3% of the Portuguese speaking population.This work focus on the ETD collection of a university in Brazil. It has offered graduate programs since the early 1960s, when T&D were published in pt, unless a special request was submitted. ETDs had been optional since 2000 and became mandatory in 2002. In 2008, ETDs presented in foreign languages started being accepted without special authorization. The current numbers of ETDs by language are:
      02.Objectives
      The objective of this work is to examine if the patterns of accesses (numbers of accesses and the numbers and languages of the countries where they originated) are different for ETDs in en and in pt.
      
      03.Method:This work is based on the analysis of data available on the Institutional Repository and organized in an OA dataset available on the same repository. Data on the dataset are numbers of ETDs and of OA ETDs, average numbers of partitions, numbers of accesses and the countries where they came from.  The dataset shows that the percentages of ETDs in each language are very different. The percentages of ETDs in en are contained in the interval from little less that 5.0 (2019) to a little over 8.0 (2023). The % of ETDs in en are growing every year, though. Examination of the statistics available on the repository shows that ETDs in en are predominantely in Science & Technology and Economics.For this reason, part of the analysis was based on averages to make up for the differences in numbers. Normalization was also used due to the fact that ETDs are partitioned and the numbers of partitions vary according to the language and to the year. Analysis is also performed on accesses from groups of countries that have either pt or en as one of the official languages. The work contains tables and graphics.
      
      04.Results and Conclusions:Most results are not surprising. The first is that accesses from Brazil account for almost 75% to ETDs in pt. The second is that accesses from the US are over 39% to ETDs in en and accesses from countries that have neither language as official are over 35% and over 9%, respectively, for en and pt. The numbers of countries are between 150 and 172 for en and between 194 and 215 for pt; it is important to remark that the numbers of ETDs in pt are over 10 times as big as the corresponding (per year) in en. A surprising result is the number of accesses from pt-speaking countries which were less than 7% for ETDs in pt. A not so surprising result is that in the five years of observation the average numbers of accesses are different for both languages but ETDs in pt had two to three times the average numbers of accesses when compared to the works in en.  There are differences in the patterns of accesses in the two sets. The percentages of accesses from Brazil are significant even to works in en. This is easy to explain due to the fact that the university is in Brazil. The other difference comes from the fact that ETDs in en are mostly in S&T and Economics, and there are many repositories in these areas offering works in en.

        \end{abstract}
        
