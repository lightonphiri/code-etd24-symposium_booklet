
        \begin{abstract}{International Visibility of ETDs in Portuguese and in English on a Brazilian Repository}{%
            Anna B. Pavani}{%
            Pontifícia Universidade Católica do Rio de Janeiro, Brazil}{%
            }

    Background and Introduction: Portuguese is the 9th most spoken language (4th Western language) in the world according to Berlitz (https://www.berlitz.com/blog/most-spoken-languages-world) or the 8th (4th Western according to Statista (https://www.statista.com/statistics/266808/the-most-spoken-languages-worldwide/). Both numbers indicate that Portuguese has a large number of speakers. The Comunidade de Países de Língua Portuguesa (https://www.cplp.org/) lists nine countries that have Portuguese as one of the official languages: Angola (36.7M), Brazil (216.4M), Cape Verde (0.6M), Guinea-Bissau (2.2M), Equatorial Guinea (1.7M), Mozambique (33.9M), Portugal (10.2M), São Tomé and Príncipe (0.2M), and Timor-Leste (1.4M). Their populations in 2023 were sourced from Worldometer (https://www.worldometers.info/world-population/population-by-country/). It can easily be observed that Brazil holds 71.3\% of the Portuguese-speaking population. This work focuses on the ETD collection of a university in Brazil. It has offered graduate programs since the early 1960s when theses and dissertations (T&D) were published in Portuguese, unless a special request was submitted. ETDs became optional in 2000 and mandatory in 2002. In 2008, ETDs presented in foreign languages started being accepted without special authorization. The current numbers of ETDs by language are presented.

    Objectives: The objective of this work is to examine if the patterns of accesses (numbers of accesses and the numbers and languages of the countries where they originated) are different for ETDs in English (en) and in Portuguese (pt).

    Method: This work is based on the analysis of data available on the Institutional Repository and organized in an open access (OA) dataset available on the same repository. Data in the dataset include numbers of ETDs and of OA ETDs, average numbers of partitions, numbers of accesses, and the countries where they originated. The dataset shows that the percentages of ETDs in each language are very different. The percentages of ETDs in English range from a little less than 5.0% (2019) to a little over 8.0% (2023). The percentage of ETDs in English is growing every year. Examination of the statistics available on the repository shows that ETDs in English are predominantly in Science & Technology and Economics. For this reason, part of the analysis was based on averages to account for the differences in numbers. Normalization was also employed due to the fact that ETDs are partitioned, and the number of partitions varies according to the language and year. Analysis is also performed on accesses from groups of countries that have either Portuguese or English as one of the official languages. The work contains tables and graphics.

    Results and Conclusions: Most results are not surprising. The first finding is that accesses from Brazil account for almost 75% of ETDs in Portuguese. The second finding is that accesses from the US account for over 39% of ETDs in English, and accesses from countries that have neither language as official are over 35% and over 9%, respectively, for English and Portuguese. The number of countries accessing ETDs ranges between 150 and 172 for English and between 194 and 215 for Portuguese. It is important to note that the number of ETDs in Portuguese is over 10 times greater than the corresponding yearly numbers in English. A surprising result is the number of accesses from Portuguese-speaking countries, which was less than 7% for ETDs in Portuguese. A less surprising result is that, in the five years of observation, the average numbers of accesses differ for both languages, with ETDs in Portuguese having two to three times the average number of accesses compared to the works in English. There are distinct patterns of access between the two sets. The significant percentage of accesses from Brazil is easy to explain due to the university's location. The other difference arises from the fact that ETDs in English are mostly in Science & Technology and Economics, fields where there are many repositories offering works in English.

\end{abstract}

